%%%%%%%%%%%%%%%%%%%%%%%%%%%%%%%%%%%%%%%%
% Wilson Resume/CV
% XeLaTeX Template
% Version 1.0 (22/1/2015)
%
% This template has been downloaded from:
% http://www.LaTeXTemplates.com
%
% Original author:
% Howard Wilson (https://github.com/watsonbox/cv_template_2004) with
% extensive modifications by Vel (vel@latextemplates.com)
%
% License:
% CC BY-NC-SA 3.0 (http://creativecommons.org/licenses/by-nc-sa/3.0/)
%
%%%%%%%%%%%%%%%%%%%%%%%%%%%%%%%%%%%%%%%%%

%----------------------------------------------------------------------------------------
%	PACKAGES AND OTHER DOCUMENT CONFIGURATIONS
%----------------------------------------------------------------------------------------

\documentclass[10pt]{article} % Default font size

\usepackage{tabularx}

%%%%%%%%%%%%%%%%%%%%%%%%%%%%%%%%%%%%%%%%%
% Wilson Resume/CV
% Structure Specification File
% Version 1.0 (22/1/2015)
%
% This file has been downloaded from:
% http://www.LaTeXTemplates.com
%
% License:
% CC BY-NC-SA 3.0 (http://creativecommons.org/licenses/by-nc-sa/3.0/)
%
%%%%%%%%%%%%%%%%%%%%%%%%%%%%%%%%%%%%%%%%%

%----------------------------------------------------------------------------------------
%	PACKAGES AND OTHER DOCUMENT CONFIGURATIONS
%----------------------------------------------------------------------------------------

\usepackage[a4paper, hmargin=25mm, vmargin=30mm, top=20mm]{geometry} % Use A4 paper and set margins

\usepackage{fancyhdr} % Customize the header and footer

\usepackage{lastpage} % Required for calculating the number of pages in the document

\usepackage[svgnames]{xcolor} 
\definecolor{Link}{RGB}{0 51 127}
\usepackage[colorlinks=true,urlcolor=Link]{hyperref} % Colors for links, text and headings

\setcounter{secnumdepth}{0} % Suppress section numbering

%\usepackage[proportional,scaled=1.064]{erewhon} % Use the Erewhon font
%\usepackage[erewhon,vvarbb,bigdelims]{newtxmath} % Use the Erewhon font
\usepackage[utf8]{inputenc} % Required for inputting international characters
\usepackage[T1]{fontenc} % Output font encoding for international characters

\usepackage{fontspec} % Required for specification of custom fonts
\setmainfont[Path = ./fonts/,
Extension = .otf,
BoldFont = Erewhon-Bold,
ItalicFont = Erewhon-Italic,
BoldItalicFont = Erewhon-BoldItalic,
SmallCapsFeatures = {Letters = SmallCaps}
]{Erewhon-Regular}

\usepackage{color} % Required for custom colors
\definecolor{slateblue}{rgb}{0.17,0.22,0.34}

\usepackage{sectsty} % Allows customization of titles
\sectionfont{\color{slateblue}} % Color section titles

\fancypagestyle{plain}{\fancyhf{}\cfoot{\thepage\ of \pageref*{LastPage}}} % Define a custom page style
\pagestyle{plain} % Use the custom page style through the document
\renewcommand{\headrulewidth}{0pt} % Disable the default header rule
\renewcommand{\footrulewidth}{0pt} % Disable the default footer rule

\setlength\parindent{0pt} % Stop paragraph indentation

% Non-indenting itemize
\newenvironment{itemize-noindent}
{\setlength{\leftmargini}{1em}\begin{itemize}}
{\end{itemize}}

% Text width for tabbing environments
\newlength{\smallertextwidth}
\setlength{\smallertextwidth}{\textwidth}
\addtolength{\smallertextwidth}{-2cm}

\newcommand{\sqbullet}{~\vrule height 1ex width .8ex depth -.2ex} % Custom square bullet point definition

%----------------------------------------------------------------------------------------
%	MAIN HEADER COMMAND
%----------------------------------------------------------------------------------------

\renewcommand{\title}[1]{
{\huge{\color{slateblue}\textbf{#1}}}\\ % Header section name and color
\rule{\textwidth}{0.5mm}\\ % Rule under the header
}

%----------------------------------------------------------------------------------------
%	JOB COMMAND
%----------------------------------------------------------------------------------------

\newcommand{\job}[5]{
\begin{tabbing}
\hspace{2.5cm} \= \kill
#1 \> \textbf{#3}, #4 \\
#2 \>\+ #5
\end{tabbing}
%\vspace{2mm}
}

%----------------------------------------------------------------------------------------
%	PROJECT COMMAND
%----------------------------------------------------------------------------------------

\newcommand{\proj}[5]{
\begin{tabbing}
\hspace{2.5cm} \= \kill
\textbf{#1} \> \href{#2}{#3} (#4)
\begin{minipage}{\smallertextwidth}
%\vspace{2mm}   
#5
\end{minipage}
\end{tabbing}
%\vspace{2mm}
}  

%----------------------------------------------------------------------------------------
%	SKILL GROUP COMMAND
%----------------------------------------------------------------------------------------

\newcommand{\skillgroup}[2]{
\begin{tabbing}
\hspace{5mm} \= \kill
\sqbullet \>\+ \textbf{#1} \\
\begin{minipage}{\smallertextwidth}
\vspace{2mm}
#2
\end{minipage}
\end{tabbing}
}

%----------------------------------------------------------------------------------------
%	INTERESTS GROUP COMMAND
%-----------------------------------------------------------------------------------------

\newcommand{\interestsgroup}[1]{
\begin{tabbing}
\hspace{5mm} \= \kill
#1
\end{tabbing}
\vspace{-10mm}
}

\newcommand{\interest}[1]{\sqbullet \> \textbf{#1}\\[3pt]} % Define a custom command for individual interests

%----------------------------------------------------------------------------------------
%	TABBED BLOCK COMMAND
%----------------------------------------------------------------------------------------

\newcommand{\tabbedblock}[1]{
\begin{tabbing}
\hspace{2.5cm} \= \hspace{4cm} \= \kill
#1
\end{tabbing}
}
 % Include the file specifying document layout

%----------------------------------------------------------------------------------------

\begin{document}

%----------------------------------------------------------------------------------------
%	NAME AND CONTACT INFORMATION
%----------------------------------------------------------------------------------------

\title{Tanakorn Leesatapornwongsa} % Print the main header

%------------------------------------------------

\parbox{0.4\textwidth}{ % First block
\begin{tabbing} % Enables tabbing
\hspace{1.5cm} \= \hspace{2cm} \= \kill % Spacing within the block
{\bf Address} \> University of Chicago\\ % Address line 1
\> Department of Computer Science \\ % Address line 2
\> Ryerson Hall, Chicago, IL 60637 \\ % Address line 2
%{\bf Date of Birth} \> 7$^{th}$ September 1979 \\ % Date of birth 
%{\bf Nationality} \> British % Nationality
\end{tabbing}}
\hfill % Horizontal space between the two blocks
\parbox{0.6\textwidth}{ % Second block
\begin{tabbing} % Enables tabbing
\hspace{2.3cm} \= \hspace{2cm} \= \kill % Spacing within the block
{\bf Web} \> \href{http://people.cs.uchicago.edu/\~tanakorn}{http://people.cs.uchicago.edu/\textasciitilde tanakorn} \\ % Web
{\bf Email} \> \href{mailto:tanakorn@cs.uchicago.edu}{tanakorn@cs.uchicago.edu} \\ % Email address
{\bf Office Phone} \> +1 (224) 256 3116 \\ % Office phone
%{\bf Mobile Phone} \> +1 (224) 256 3116 \\ % Mobile phone
\end{tabbing}}

%----------------------------------------------------------------------------------------
%	PERSONAL PROFILE
%----------------------------------------------------------------------------------------

%\section{Personal Profile}

%----------------------------------------------------------------------------------------
%	EDUCATION SECTION
%----------------------------------------------------------------------------------------

\section{Education}

\tabbedblock{
2012 - Present \> \textbf{University of Chicago}, Chicago, IL, USA\\
\>\textit{Ph.D. in Computer Sciences}, expected September 2017\\
\>Advisor: Prof. Haryadi S. Gunawi
}

%------------------------------------------------

\tabbedblock{
2005 - 2009 \> \textbf{Chulalongkorn University}, Bangkok, Thailand\\
\>\textit{B.Eng. in Computer Engineering}
}

%----------------------------------------------------------------------------------------
%	RESEARCH INTERESTS
%----------------------------------------------------------------------------------------

\section{Research Interests}

\begin{tabbing}
\hspace{2.5cm} \= \kill
\textbf{Areas} \> Operating Systems, Distributed Systems, Cloud Computing, and Storage Systems. \\
\textbf{Focuses} \> Reliability and Scalability.
\end{tabbing}

%----------------------------------------------------------------------------------------
%	EMPLOYMENT HISTORY SECTION
%----------------------------------------------------------------------------------------

\section{Work Experiences}

\job{Apr 2013 -}{Present}{Research Assistant}
{University of Chicago, Chicago, IL, USA}
{\begin{minipage}{\smallertextwidth}
Working on \dblquote{\textit{improving the dependability of distributed systems and cloud systems}} topic
with \href{http://ucare.cs.uchicago.edu/}{Prof. Haryadi S. Gunawi}
in \href{http://ucare.cs.uchicago.edu/}{UCARE} Group
\end{minipage}} 

\job{Jun 2016 -}{Aug 2016}{Intern}
{Microsoft Research, Redmond, WA, USA}
{\begin{minipage}{\smallertextwidth}
Worked on \dblquote{\textit{distributed systems reliability testing}} project,
with \href{https://www.microsoft.com/en-us/research/people/madanm/}{Madan Musuvathi}, 
\href{https://www.microsoft.com/en-us/research/people/sumann/}{Suman Nath}, 
and \href{http://people.csail.mit.edu/lenin/}{Lenin Ravindranath Sivalingam}
\end{minipage}} 

\job{Oct 2015}{}{Guest Lecturer}
{University of Chicago, Chicago, IL, USA}
{Guest lecturer in Advanced Operating Systems course} 

\job{Jun 2014 -}{Aug 2014}{Intern}
{NetApp, Inc., Sunnyvale, CA, USA}
{\begin{minipage}{\smallertextwidth}
Worked on \dblquote{\textit{distributed systems scalability checking}} project, 
with Gokul Soundararajan
in \href{http://www.netapp.com/us/company/leadership/advanced-technology/}{Advanced Technology Group (ATG)} 
\end{minipage}}

\job{Oct 2012 -}{Mar 2013}{Teaching Assistant}
{University of Chicago, Chicago, IL, USA}
{TA in Computer Architecture and Mobile Computing courses} 

\job{2009 - 2012}{}{Platform Engineer}
{Wavify Inc., Bangkok, Thailand}
{\begin{minipage}{\smallertextwidth}
\begin{itemize-noindent}
\setlength\itemsep{-1ex}
\item Built P2P data synchronization framework for mobile platform
\item Built file offloading network-attached storage (NAS) for high workload mail server
\end{itemize-noindent}
\end{minipage}}

%----------------------------------------------------------------------------------------
%	PUBLICATIONS
%----------------------------------------------------------------------------------------

\section{Publications}
\underline{Tanakorn Leesatapornwongsa}, Jeffrey F. Lukman, Shan Lu, and Haryadi
S. Gunawi (the two student authors performed equal work, I was the presenter)
\textbf{TaxDC: A Taxonomy of Non-Deterministic Concurrency Bugs in Datacenter
Distributed Systems}. In \textit{Proceedings of Architectural Support for
Programming Languages and Operating Systems (ASPLOS)}, 2016 
\vspace{2mm}

\underline{Tanakorn Leesatapornwongsa}, and Haryadi S. Gunawi. \textbf{SAMC: A
Fast Model Checker for Finding Heisenbugs in Distributed Systems}. In
\textit{Proceedings of International Symposium on Software Testing and Analysis
(ISSTA)}, 2015
\vspace{2mm}

\underline{Tanakorn Leesatapornwongsa}, Mingzhe Hao, Pallavi Joshi, Jeffrey F.
Lukman, and Haryadi S. Gunawi. \textbf{SAMC: Semantic-Aware Model Checking for
Fast Discovery of Deep Bugs in Cloud Systems}. In \textit{Proceedings of the
11th USENIX Symposium on Operating Systems Design and Implementation (OSDI)},
2014
\vspace{2mm}

\newpage

Haryadi S. Gunawi, Mingzhe Hao, \underline{Tanakorn Leesatapornwongsa}, Tiratat
Patana-anake, Thanh Do, Jeffry Adityatama, Kurnia J. Eliazar, Agung Laksono,
Jeffrey F. Lukman, Vincentius Martin, and Anang D. Satria (Instituitional Order).
\textbf{What Bugs Live in the Cloud? A Study of 3000+ Issues in Cloud Systems}.
In \textit{Proceedings of the 5th ACM Symposium on Cloud Computing (SoCC)}, 2014
\vspace{2mm}

\underline{Tanakorn Leesatapornwongsa} and Haryadi S. Gunawi. \textbf{The Case
for Drill-Ready Cloud Computing}. In \textit{Proceedings of the 5th ACM
Symposium on Cloud Computing (SoCC)}, 2014
\vspace{2mm}

Thanh Do, Mingzhe Hao, \underline{Tanakorn Leesatapornwongsa}, Tiratat
Patana-anake, and Haryadi S. Gunawi (Student Names are in Alphabetical Order).
\textbf{Limplock: Understanding the Impact of Limpware on Scale-Out Cloud
Systems}. In \textit{Proceedings of the 4th ACM Symposium on Cloud Computing
(SoCC)}, 2013

%----------------------------------------------------------------------------------------
%	TALKS SECTION
%----------------------------------------------------------------------------------------

\section{Talks}

\begin{tabbing}
\hspace{2.5cm} \= \kill
Apr 2016 \> \textbf{TaxDC: A Taxonomy of Non-Deterministic Concurrency Bugs in Datacenter Distributed Systems}. \\
\> (ASPLOS '16)
\end{tabbing}

\begin{tabbing}
\hspace{2.5cm} \= \kill
Jul 2015 \> \textbf{SAMC: A Fast Model Checker for Finding Heisenbugs in Distributed System}. (ISSTA '15)
\end{tabbing}

\begin{tabbing}
\hspace{2.5cm} \= \kill
Nov 2014 \> \textbf{The Case for Drill-Ready Cloud Computing}. (SoCC '14)
\end{tabbing}

\begin{tabbing}
\hspace{2.5cm} \= \kill
Oct 2014 \> \textbf{SAMC: Semantic-Aware Model Checking for Fast Discovery of Deep Bugs in Cloud Systems}. \\
\> (OSDI '14)
\end{tabbing}

%----------------------------------------------------------------------------------------
%	POSTER SECTION
%----------------------------------------------------------------------------------------

\section{Posters}

\underline{Tanakorn Leesatapornwongsa,} Jeffrey F. Lukman, Shan Lu, and Haryadi
S. Gunawi. \textbf{TaxDC: A Taxonomy of Non-Deterministic Concurrency Bugs in
Datacenter Distributed Systems}. In \textit{Proceedings of Architectural
Support for Programming Languages and Operating Systems (ASPLOS)}, 2016
\vspace{2mm}

Haryadi S. Gunawi, Mingzhe Hao, \underline{Tanakorn Leesatapornwongsa}, Tiratat
Patana-anake, Thanh Do, Jeffry Adityatama, Kurnia J. Eliazar, Agung Laksono,
Jeffrey F. Lukman, Vincentius Martin, and Anang D. Satria. \textbf{What Bugs
Live in the Cloud? A Study of 3000+ Issues in Cloud Systems}. In
\textit{Proceedings of the 5th ACM Symposium on Cloud Computing (SoCC)}, 2014
\vspace{2mm}

\underline{Tanakorn Leesatapornwongsa,} and Haryadi S. Gunawi. \textbf{The Case
for Drill-Ready Cloud Computing}. In \textit{Proceedings of the 5th ACM
Symposium on Cloud Computing (SoCC)}, 2014
\vspace{2mm}

\underline{Tanakorn Leesatapornwongsa,} Mingzhe Hao, Pallavi Joshi, Jeffrey F.
Lukman, and Haryadi S. Gunawi. \textbf{SAMC: Semantic-Aware Model Checking for
Fast Discovery of Deep Bugs in Cloud Systems}. In \textit{Proceedings of the
11th USENIX Symposium on Operating Systems Design and Implementation (OSDI)},
2014
\vspace{2mm}

Thanh Do, Mingzhe Hao, \underline{Tanakorn Leesatapornwongsa}, Tiratat
Patana-anake, and Haryadi S. Gunawi. \textbf{Limplock: Understanding the Impact
of Limpware on Scale-Out Cloud Systems}. In \textit{Proceedings of the 4th ACM
Symposium on Cloud Computing (SoCC)}, 2013

%----------------------------------------------------------------------------------------
%	AWARD SECTION
%----------------------------------------------------------------------------------------

\section{Awards and Honors}

\begin{tabbing}
\hspace{2.5cm} \= \kill
2016 \> \textbf{2016 - 2017 Facebook Fellowship Finalist}, Facebook
\end{tabbing}

\begin{tabbing}
\hspace{2.5cm} \= \kill
2015 \> \textbf{ISSTA '15 Student Financial Support}, US National Science Foundation (NSF)
\end{tabbing}

\begin{tabbing}
\hspace{2.5cm} \= \kill
2014 \> \textbf{SoCC '14 Student Scholarship}, Association for Computing Machinery (ACM)
\end{tabbing}

\begin{tabbing}
\hspace{2.5cm} \= \kill
2014 \> \textbf{OSDI '14 Student Grant}, USENIX
\end{tabbing}

\begin{tabbing}
\hspace{2.5cm} \= \kill
2014 \> \textbf{UU Fellowship}, University of Chicago
\end{tabbing}

\begin{tabbing}
\hspace{2.5cm} \= \kill
2009 \> \textbf{2nd Class Honor}, Computer Engineering Department, Chulalongkorn University
\end{tabbing}

\begin{tabbing}
\hspace{2.5cm} \= \kill
2008 \> \textbf{1st Place World RoboCup Championship}, RoboCup Soccer Small Size League
\end{tabbing}

\pagebreak

%----------------------------------------------------------------------------------------
%	PROJECT SECTION
%----------------------------------------------------------------------------------------

\section{Projects}
\vspace{-4mm}
{\footnotesize Please click on the titles for the reports or more information}

\subsection{Research Projects}

\begin{tabbing}
\hspace{2.5cm} \= \kill
2014 - Present \>\+ \textbf{Scalability-Checkable Cloud Systems} (Ongoing project) \\
\begin{minipage}{\smallertextwidth}
I am building scalability-checkable cloud systems, the systems in which testers
can detect problems of scalability in convenient manner and at an afforadable
cost. In this work, I build a scalability simulator to simulate behaviors of the
systems when they are running at large scale. The technique can be applied to
all types of cloud systems such as distributed file systems, distributed
storage, or distributed computing framework.
\end{minipage}
\end{tabbing}

\begin{tabbing}
\hspace{2.5cm} \= \kill
2015 \>\+ \href{http://ucare.cs.uchicago.edu/pdf/asplos16-TaxDC.pdf}{\textbf{TaxDC: A Comprehensive Taxonomy of Heisenbugs in Cloud Distributed Systems}} (\textit{ASPLOS '16}) \\
\begin{minipage}{\smallertextwidth}
I as one of the project leaders, studied 104 distributed concurrency bugs (DC
bugs) from four widely-deployed cloud systems, Cassandra, Hadoop MapReduce,
HBase, and ZooKeeper. The study covers DC-bug characteristics along several axes
of analysis such as the triggering condition and input preconditions, failure
symptoms, and fix strategies, collectively stored as 2,083 classification labels.
This study is the largest and most comprehensive taxonomy of DC bugs in cloud
systems.
\end{minipage}
\end{tabbing}

\begin{tabbing}
\hspace{2.5cm} \= \kill
2013 - 2014 \> \href{http://ucare.cs.uchicago.edu/pdf/osdi14-samc.pdf}{\textbf{SAMC: Semantic-Aware Model Checking for Fast Discovery of Deep Bugs in Cloud Systems}} \\
\>\+ (\textit{OSDI '14} and \textit{ISSTA '15}) \\
\begin{minipage}{\smallertextwidth}
I introduced semantic-aware model checking (SAMC), a white-box principle that
takes simple semantic information of the target system and incorporates the
knowledge into state-space reduction policies. I built the prototype of SAMC
from scratch for a total of 10,886 lines of code, and integrated it to three
cloud systems, Cassandra, Hadoop MapReduce, and ZooKeeper. SAMC can reproduce
old bugs beneath deeply in these systems faster than other reduction policies up
to 271x (33x on average), and it can find two new bugs in these systems.
\end{minipage}
\end{tabbing}

\begin{tabbing}
\hspace{2.5cm} \= \kill
2014 \>\+ \href{http://ucare.cs.uchicago.edu/pdf/socc14-drill.pdf}{\textbf{The Case for Drill-Ready Cloud Computing}} (\textit{SoCC '14}) \\
\begin{minipage}{\smallertextwidth}
In this work, I explored the fundamental question: \dblquote{\textit{how can we ensure
that cloud services work robustly against many failure scenarios in real
deployments?}}. And to further the current answers, I proposed a vision of a new
reliability paradigm, the Drill-Ready Cloud Computing. The online testing
framework that provides safety, efficiency, usability, and generality for
testing cloud systems.
\end{minipage}
\end{tabbing}

\begin{tabbing}
\hspace{2.5cm} \= \kill
2013 - 2014 \>\+ \href{http://ucare.cs.uchicago.edu/pdf/socc14-cbs.pdf}{\textbf{What Bugs Live in the Cloud? A Study of 3000+ Issues in Cloud Systems}} (\textit{SoCC '14}) \\
\begin{minipage}{\smallertextwidth}
Our group and I comprehensively studied the issues in development and deployment of six
popular cloud systems, Hadoop MapReduce, HDFS, HBase, Cassandra, ZooKeeper and
Flume. We reviewed in total 21,399 submitted issues within a three-year period
(2011- 2014), and performed a deep analysis of 3,655 \dblquote{\textit{vital}} issues among these
issues with a set of detailed classifications. We also derived numerous
interesting insights unique to cloud systems. 
\end{minipage}
\end{tabbing}

\begin{tabbing}
\hspace{2.5cm} \= \kill
2013 \>\+ \href{http://ucare.cs.uchicago.edu/pdf/socc13-limplock.pdf}{\textbf{Limplock: Understanding the Impact of Limpware on Scale-Out Cloud Systems}} (\textit{SoCC 2013}) \\
%\>\+ (\textit{SoCC 2013}) \\
\begin{minipage}{\smallertextwidth}
Our group and I studied one often-overlooked cause of performance failure:
"\textit{limpware}", limping hardware whose performance degrades significantly
compares to its specification. We assembled a set of benchmarks that combine
data-intensive load and limpware injections to show the impact of limpware on
five cloud systems, Cassandra, Hadoop, HBase, HDFS, and ZooKeeper. We
also unearthed why the systems cannot tolerate limpware.
\end{minipage}
\end{tabbing}

\subsection{Industry Projects}

\begin{tabbing}
\hspace{2.5cm} \= \kill
2011 - 2012 \>\+ \textbf{Crossweaver: Data Synchronization Framework for Mobile Platform} (\textit{Wavify Inc.}) \\
\begin{minipage}{\smallertextwidth}
I developed a P2P framework for mobile applications to synchronize data with
each other with the notion of versioning and access control, or synchronize
back to Wavify NextStor, Wavify's cloud storage, for backup.
\end{minipage}
\end{tabbing}

\begin{tabbing}
\hspace{2.5cm} \= \kill
2010 - 2011 \>\+ \textbf{Wavify NextStor} (\textit{Wavify Inc.}) \\
\begin{minipage}{\smallertextwidth}
I built a cloud storage appliance for offloading users' files from Wavify's
mail server product. It also can be employed for enterprise multi-purpose
private cloud storage, with an affordable, scalable storage options.
\end{minipage}
\end{tabbing}

%----------------------------------------------------------------------------------------
%	SERVICES SECTION
%----------------------------------------------------------------------------------------

\newpage

\section{Professional Service}

\begin{tabbing}
\hspace{2.5cm} \= \hspace{3.5cm}  \= \kill
2015 \> \textbf{External reviewer} \> \textit{FAST '16}, USENIX Conference on File and Storage Technologies
\end{tabbing}

%----------------------------------------------------------------------------------------
%	ADVISING SECTION
%----------------------------------------------------------------------------------------

\section{Advising}

I co-advise the following students along with my advisor. I meet with them
every week to direct them in research. Some are students in Univeristy at
Chicago, and some are remote students in Indonesia.
\begin{tabbing}
\hspace{3.5cm} \= \hspace{2.5cm} \= \kill
\textbf{UChicago PhD} \> (2 student) \> Jeffrey F. Lukman, and Huan Ke\\
\textbf{UChicago Masters} \> (4 students) \> Bo Fu, Murphy Zhang, Yanzhe Wu, and Cesar Studardo\\
\textbf{Remote Students} \> (4 students) \> Dikaimin Simon (Surya University),\\
\> \>Danial Heri Kurniawan and Satria Priambada (Bandung Institute of Technology)\\
\> \>Khoirul Hasin (Sepuluh Nopember Institute of Technology)
\end{tabbing}

%----------------------------------------------------------------------------------------
%	TECHNICAL SKILL SECTION
%----------------------------------------------------------------------------------------

\section{Technical Skills}

\begin{tabbing}
\hspace{4cm} \= \kill
\textbf{Operating Systems} \> FreeBSD, Linux \\
\textbf{Distributed Systems} \> Hadoop, ZooKeeper, Cassandra \\
\textbf{File/Storage Systems} \> ext3, RAID, HDFS, Azure Storage, Cleversafe Storage Cluster\\
\textbf{Programming} \> Java, Python, C/C++, C\texttt{\#}
\end{tabbing}

\end{document}
