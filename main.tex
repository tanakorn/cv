%%%%%%%%%%%%%%%%%%%%%%%%%%%%%%%%%%%%%%%%%
% Wilson Resume/CV
% XeLaTeX Template
% Version 1.0 (22/1/2015)
%
% This template has been downloaded from:
% http://www.LaTeXTemplates.com
%
% Original author:
% Howard Wilson (https://github.com/watsonbox/cv_template_2004) with
% extensive modifications by Vel (vel@latextemplates.com)
%
% License:
% CC BY-NC-SA 3.0 (http://creativecommons.org/licenses/by-nc-sa/3.0/)
%
%%%%%%%%%%%%%%%%%%%%%%%%%%%%%%%%%%%%%%%%%

%----------------------------------------------------------------------------------------
%	PACKAGES AND OTHER DOCUMENT CONFIGURATIONS
%----------------------------------------------------------------------------------------

\documentclass[10pt]{article} % Default font size

\usepackage{tabularx}

\input{structure.tex} % Include the file specifying document layout

%----------------------------------------------------------------------------------------

\begin{document}

%----------------------------------------------------------------------------------------
%	NAME AND CONTACT INFORMATION
%----------------------------------------------------------------------------------------

\title{Tanakorn Leesatapornwongsa} % Print the main header

%------------------------------------------------

\parbox{0.4\textwidth}{ % First block
\begin{tabbing} % Enables tabbing
\hspace{1.5cm} \= \hspace{2cm} \= \kill % Spacing within the block
{\bf Address} \> University of Chicago\\ % Address line 1
\> Department of Computer Science \\ % Address line 2
\> Ryerson Hall, Chicago, IL 60637 \\ % Address line 2
%{\bf Date of Birth} \> 7$^{th}$ September 1979 \\ % Date of birth 
%{\bf Nationality} \> British % Nationality
\end{tabbing}}
\hfill % Horizontal space between the two blocks
\parbox{0.6\textwidth}{ % Second block
\begin{tabbing} % Enables tabbing
\hspace{2.3cm} \= \hspace{2cm} \= \kill % Spacing within the block
{\bf Web} \> \href{http://people.cs.uchicago.edu/\~tanakorn}{http://people.cs.uchicago.edu/\textasciitilde tanakorn} \\ % Web
{\bf Email} \> \href{mailto:tanakorn@cs.uchicago.edu}{tanakorn@cs.uchicago.edu} \\ % Email address
{\bf Office Phone} \> +1 (773) 702 6614 \\ % Office phone
%{\bf Mobile Phone} \> +1 (224) 256 3116 \\ % Mobile phone
\end{tabbing}}

%----------------------------------------------------------------------------------------
%	PERSONAL PROFILE
%----------------------------------------------------------------------------------------

%\section{Personal Profile}

%Lorem ipsum dolor sit amet, consectetur adipiscing elit. Duis elementum nec dolor sed sagittis. Cras justo lorem, volutpat mattis lacus vel, consequat aliquam quam. Interdum et malesuada fames ac ante ipsum primis in faucibus. Integer blandit, massa at tincidunt ornare, dolor magna interdum felis, ac blandit urna neque in turpis.

%----------------------------------------------------------------------------------------
%	EDUCATION SECTION
%----------------------------------------------------------------------------------------

\section{Education}

\tabbedblock{
2012 - Present \> \textbf{University of Chicago}, Chicago, IL, USA\\
\>\textit{Ph.D. in Computer Sciences}, expected September 2018\\
\>Advisor: Prof. Haryadi S. Gunawi
}

%------------------------------------------------

\tabbedblock{
2005 - 2009 \> \textbf{Chulalongkorn University}, Bangkok, Thailand\\
\>\textit{B.Eng. in Computer Engineering}
}

%----------------------------------------------------------------------------------------
%	APPLICABLE COURSEWORK
%----------------------------------------------------------------------------------------

\section{Applicable Coursework}

\begin{tabularx}{0.8\textwidth}{ X X }
 Discrete Mathematics & Algorithms  \\
 Machine Learning & Computational Linguistics \\
 Computer Architecture & Advanced Operating Systems  \\
 Data-Intensive Computing & Cloud Computing  \\
\end{tabularx}

%----------------------------------------------------------------------------------------
%	RESEARCH INTERESTS
%----------------------------------------------------------------------------------------

\section{Research Interests}

\begin{tabbing}
\hspace{2.5cm} \= \kill
\textbf{Areas} \> Operating Systems, Storage Systems, Distributed Systems, and Cloud Computing. \\
\textbf{Focuses} \> Reliability and Scalability.
\end{tabbing}

%----------------------------------------------------------------------------------------
%	EMPLOYMENT HISTORY SECTION
%----------------------------------------------------------------------------------------

\section{Work Experiences}

\if 0
\job{Oct 2015}{}{Guest Lecturer}
{University of Chicago, Chicago, IL, USA}
{Guest lecturer in Advanced Operating Systems course} 
\fi

\job{Apr 2013 -}{Present}{Research Assistant}
{University of Chicago, Chicago, IL, USA}
{Work with Prof. Haryadi S. Gunawi in UCARE Group} 

\job{Jun 2014 -}{Aug 2014}{Intern}
{NetApp, Inc., Sunnyvale, CA, USA}
{Worked in Advanced Technology Group (ATG)} 

\job{Oct 2012 -}{Mar 2013}{Teaching Assistant}
{University of Chicago, Chicago, IL, USA}
{TA in Computer Architecture and Mobile Computing courses} 

\job{2009 - 2012}{}{Platform Engineer}
{Wavify Inc., Bangkok Thailand}
{\begin{minipage}{\smallertextwidth}
\begin{itemize-noindent}
\setlength\itemsep{-1ex}
\item Built data synchronization framework for mobile platform
\item Built file offloading NAS for high workload mail server
\end{itemize-noindent}
\end{minipage}}

%----------------------------------------------------------------------------------------
%	PUBLICATIONS
%----------------------------------------------------------------------------------------

\section{Publications}

Haryadi S. Gunawi, \underline{Tanakorn Leesatapornwongsa}, Shan Lu, and Jeffrey
Lukman (Alphabetical Order). \textbf{TaxDC: A Comprehensive Taxonomy of
Heisenbugs in Cloud Distributed Systems}. In \textit{Submission to
Architectural Support for Programming Languages and Operating Systems
(ASPLOS)}, 2016 
\vspace{2mm}

\underline{Tanakorn Leesatapornwongsa}, and Haryadi S. Gunawi. \textbf{SAMC: A
Fast Model Checker for Finding Heisenbugs in Distributed Systems}. In
\textit{Proceedings of International Symposium on Software Testing and Analysis
(ISSTA)}, 2015
\vspace{2mm}

\underline{Tanakorn Leesatapornwongsa}, Mingzhe Hao, Pallavi Joshi, Jeffrey F.
Lukman, and Haryadi S. Gunawi. \textbf{SAMC: Semantic-Aware Model Checking for
Fast Discovery of Deep Bugs in Cloud Systems}. In \textit{Proceedings of the
11th USENIX Symposium on Operating Systems Design and Implementation (OSDI)},
2014
\vspace{2mm}

Haryadi S. Gunawi, Mingzhe Hao, \underline{Tanakorn Leesatapornwongsa}, Tiratat
Patana-anake, Thanh Do, Jeffry Adityatama, Kurnia J. Eliazar, Agung Laksono,
Jeffrey F. Lukman, Vincentius Martin, and Anang D. Satria (Instituitional Order).
\textbf{What Bugs Live in the Cloud? A Study of 3000+ Issues in Cloud Systems}.
In \textit{Proceedings of the 5th ACM Symposium on Cloud Computing (SoCC)}, 2014
\vspace{2mm}

\underline{Tanakorn Leesatapornwongsa} and Haryadi S. Gunawi. \textbf{The Case
for Drill-Ready Cloud Computing}. In \textit{Proceedings of the 5th ACM
Symposium on Cloud Computing (SoCC)}, 2014
\vspace{2mm}

Thanh Do, Mingzhe Hao, \underline{Tanakorn Leesatapornwongsa}, Tiratat
Patana-anake, and Haryadi S. Gunawi (Student Names are in Alphabetical Order).
\textbf{Limplock: Understanding the Impact of Limpware on Scale-Out Cloud
Systems}. In \textit{Proceedings of the 4th ACM Symposium on Cloud Computing
(SoCC)}, 2013

%----------------------------------------------------------------------------------------
%	TALKS SECTION
%----------------------------------------------------------------------------------------

\section{Talks}

\begin{tabbing}
\hspace{2.5cm} \= \kill
Jul 2015 \> \textbf{SAMC: A Fast Model Checker for Finding Heisenbugs in Distributed System}. \\
\> (ISSTA '15)
\end{tabbing}

\begin{tabbing}
\hspace{2.5cm} \= \kill
Nov 2014 \> \textbf{The Case for Drill-Ready Cloud Computing}. (SoCC '14)
\end{tabbing}

\begin{tabbing}
\hspace{2.5cm} \= \kill
Oct 2014 \> \textbf{SAMC: Semantic-Aware Model Checking for Fast Discovery of Deep Bugs in 
Cloud} \\
\> \textbf{Systems}. (OSDI '14)
\end{tabbing}

%----------------------------------------------------------------------------------------
%	POSTER SECTION
%----------------------------------------------------------------------------------------

\section{Posters}

\underline{Tanakorn Leesatapornwongsa,} and Haryadi S. Gunawi. \textbf{The Case
for Drill-Ready Cloud Computing}. In \textit{Proceedings of the 5th ACM
Symposium on Cloud Computing (SoCC)}, 2014
\vspace{2mm}

\underline{Tanakorn Leesatapornwongsa,} Mingzhe Hao, Pallavi Joshi, Jeffrey F.
Lukman, and Haryadi S. Gunawi. \textbf{SAMC: Semantic-Aware Model Checking for
Fast Discovery of Deep Bugs in Cloud Systems}. In \textit{Proceedings of the
11th USENIX Symposium on Operating Systems Design and Implementation (OSDI)},
2014
\vspace{2mm}

Thanh Do, Mingzhe Hao, \underline{Tanakorn Leesatapornwongsa}, Tiratat
Patana-anake, and Haryadi S. Gunawi. \textbf{Limplock: Understanding the Impact
of Limpware on Scale-Out Cloud Systems}. In \textit{Proceedings of the 4th ACM
Symposium on Cloud Computing (SoCC)}, 2013

%----------------------------------------------------------------------------------------
%	AWARD SECTION
%----------------------------------------------------------------------------------------

\section{Awards and Honors}

\begin{tabbing}
\hspace{2.5cm} \= \kill
2015 \> \textbf{ISSTA '15 Student Financial Support}, US National Science Foundation (NSF)
\end{tabbing}

\begin{tabbing}
\hspace{2.5cm} \= \kill
2014 \> \textbf{SoCC '14 Student Scholarship}, Association for Computing Machinery (ACM)
\end{tabbing}

\begin{tabbing}
\hspace{2.5cm} \= \kill
2014 \> \textbf{OSDI '14 Student Grant}, USENIX
\end{tabbing}

\begin{tabbing}
\hspace{2.5cm} \= \kill
2014 \> \textbf{UU Fellowship}, University of Chicago
\end{tabbing}

\begin{tabbing}
\hspace{2.5cm} \= \kill
2009 \> \textbf{2nd Class Honor}, Computer Engineering Department, Chulalongkorn University
\end{tabbing}

\begin{tabbing}
\hspace{2.5cm} \= \kill
2008 \> \textbf{1st Place World RoboCup Championship}, RoboCup Soccer Small Size League
\end{tabbing}


%----------------------------------------------------------------------------------------
%	PROJECT SECTION
%----------------------------------------------------------------------------------------

\section{Projects}
\vspace{-4mm}
{\footnotesize Please click on the titles for the reports or more information}

\subsection{Research Projects}

\begin{tabbing}
\hspace{2.5cm} \= \kill
2014 - Present \>\+ \textbf{Scalability-Checkable Cloud Systems} (Ongoing project) \\
\begin{minipage}{\smallertextwidth}
I am building scalability-checkable cloud systems, the systems in which testers can detect
problems of scalability in convenient manner and at an afforadable cost. In
this work, I build a scalability simulator to simulate behaviors of the systems when they
are running at large scale. The technique can be applied to all types of cloud systems such
as distributed file systems, distributed storage, or distributed computing framework.
\end{minipage}
\end{tabbing}

\begin{tabbing}
\hspace{2.5cm} \= \kill
2015 \> \textbf{TaxDC: A Comprehensive Taxonomy of Heisenbugs in Cloud Distributed Systems} \\
\>\+ (Submitted to \textit{ASPLOS 2014}) \\
\begin{minipage}{\smallertextwidth}
We study 104 distributed concurrency bugs (DC bugs) from four 
widely-deployed cloud systems, Cassandra, Hadoop MapReduce, HBase, and
ZooKeeper. The study covers DC-bug characteristics along several axes of analysis such
as the triggering condition and input preconditions, failure symptoms, and fix strategies, 
collectively stored as 2083 classification labels Our study is the largest and most 
comprehensive taxonomy of DC bugs in cloud systems.
\end{minipage}
\end{tabbing}

\begin{tabbing}
\hspace{2.5cm} \= \kill
2013 - 2014 \> \href{http://ucare.cs.uchicago.edu/pdf/osdi14-samc.pdf}{\textbf{SAMC: Semantic-Aware Model Checking for Fast Discovery of Deep Bugs in Cloud Systems}} \\
\>\+ (\textit{OSDI 2014} and \textit{ISSTA 2015}) \\
\begin{minipage}{\smallertextwidth}
We introduce semantic-aware model checking (SAMC), a white-box principle that
takes simple semantic information of the target system and incorporates the
knowledge into state-space reduction policies. We build the prototype of SAMC
from scratch for a total of 10,886 lines of code, and 
integrate it to three cloud systems, Cassandra, Hadoop MapReduce, and ZooKeeper.
SAMC can reproduce old bugs beneath deeply in these systems faster than other
reduction policies.
\end{minipage}
\end{tabbing}

\begin{tabbing}
\hspace{2.5cm} \= \kill
2014 \>\+ \href{http://ucare.cs.uchicago.edu/pdf/socc14-drill.pdf}{\textbf{The Case for Drill-Ready Cloud Computing}} (\textit{SoCC 2014}) \\
\begin{minipage}{\smallertextwidth}
In this work, we explore the fundamental question: "\textit{how can we ensure
that cloud services work robustly against many failure scenarios in real
deployments?}" And to further the current answers, we propose a vision of a new
reliability paradigm, the Drill-Ready Cloud Computing. The online testing
framework that provides safety, efficiency, usability, and generality for
testing cloud systems.
\end{minipage}
\end{tabbing}

\begin{tabbing}
\hspace{2.5cm} \= \kill
2013 - 2014 \>\+ \href{http://ucare.cs.uchicago.edu/pdf/socc14-cbs.pdf}{\textbf{What Bugs Live in the Cloud? A Study of 3000+ Issues in Cloud Systems}} (\textit{SoCC 2014}) \\
\begin{minipage}{\smallertextwidth}
We comprehensively study the issues in development and deployment of six popular cloud
systems, Hadoop MapReduce, HDFS, HBase, Cassandra, ZooKeeper and Flume. We
review in total 21,399 submitted issues within a three-year period (2011- 2014),
and perform a deep analysis of 3655 "vital" issues among these issues with a
set of detailed classifications. We also derive numerous interesting insights
unique to cloud systems. 
\end{minipage}
\end{tabbing}

\begin{tabbing}
\hspace{2.5cm} \= \kill
2013 \> \href{http://ucare.cs.uchicago.edu/pdf/socc13-limplock.pdf}{\textbf{Limplock: Understanding the Impact of Limpware on Scale-Out Cloud Systems}} \\
\>\+ (\textit{SoCC 2013}) \\
\begin{minipage}{\smallertextwidth}
We study one often-overlooked cause of performance failure: "limpware", limping
hardware whose performance degrades significantly compares to its specification
We assembles a set of benchmarks that combine data-intensive load and limpware
injections to show the impact of limpware on five cloud systems Cassandra,
Hadoop, HBase, HDFS, and ZooKeeper. We also unearth why the systems cannot
tolerate limpware.
\end{minipage}
\end{tabbing}

\subsection{Industry Projects}

\begin{tabbing}
\hspace{2.5cm} \= \kill
2011 - 2012 \>\+ \textbf{Crossweaver: Data Synchronization Framework for Mobile Platform} (\textit{Wavify Inc.}) \\
\begin{minipage}{\smallertextwidth}
We develop a framework for mobile applications to synchronize data with
each other with the notion of versioning and access control, or synchronize back
to Wavify NextStor, Wavify's cloud storage, for backup.
\end{minipage}
\end{tabbing}

\begin{tabbing}
\hspace{2.5cm} \= \kill
2010 - 2011 \>\+ \textbf{Wavify NextStor} (\textit{Wavify Inc.}) \\
\begin{minipage}{\smallertextwidth}
We build a cloud storage appliance for offloading user file from Wavify's
mail server product. But it also can be employed for enterprise multi-purpose
private cloud storage, with an affordable, scalable storage options.
\end{minipage}
\end{tabbing}

%----------------------------------------------------------------------------------------
%	ADVISING SECTION
%----------------------------------------------------------------------------------------

\section{Advising}

I am co-advising the followning students along with my advisor. I meet with them
every week to direct them in research. Some are students in Univeristy at
Chicago, and some are remote students in Indonesia.
\begin{tabbing}
\hspace{4cm} \= \hspace{1cm} \= \kill
\textbf{UChicago PhD} \> (1) \> Jeffrey Ferrari Lukman\\
\textbf{UChicago Masters} \> (2) \> Bo Fu and Murphy Zhang\\
\textbf{UChicago Undergrad} \> (1) \> Xue Hng Chuang\\
\textbf{Remote Students} \> (4) \> Dikaimin Simon (Surya University),\\
\> \>Danial Heri Kurniawan and Satria Priambada (Bandung Institute of Technology)\\
\> \>Khoirul Hasin (Sepuluh Nopember Institute of Technology)
\end{tabbing}


%----------------------------------------------------------------------------------------
%	TECHNICAL SKILL SECTION
%----------------------------------------------------------------------------------------

\section{Technical Skills}

\begin{tabbing}
\hspace{4cm} \= \kill
\textbf{Operating Systems} \> FreeBSD, Linux \\
\textbf{Distributed Systems} \> Hadoop, ZooKeeper, Cassandra \\
\textbf{File/Storage Systems} \> ext3, RAID, HDFS, Cleversafe Storage Cluster\\
\textbf{Programming} \> Java, Python, C/C++
\end{tabbing}

\end{document}
